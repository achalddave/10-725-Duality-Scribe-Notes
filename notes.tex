\documentclass[11pt]{article}
\usepackage{amsmath,textcomp,amssymb,geometry,graphicx,hyperref,mathtools,textcomp}
\usepackage[final]{pdfpages}
\usepackage[export]{adjustbox}
\usepackage{enumerate}
\usepackage{titlesec}
\usepackage{listings}
\usepackage[usenames, dvipsnames]{color}
\usepackage{cleveref}

\lstdefinestyle{customc}{
  belowcaptionskip=1\baselineskip,
  breaklines=true,
  frame=L,
  xleftmargin=\parindent,
  language=C,
  showstringspaces=false,
  keywordstyle=\bfseries\color{green!40!black},
  commentstyle=\itshape\color{purple!40!black},
  identifierstyle=\color{blue},
  stringstyle=\color{orange},
}

\lstset{escapechar=@,style=customc}

\def\Name{Achal Dave}
\def\CourseName{10-725}
\def\CourseSemester{Fall}
\def\CourseYear{2016}

\title{Scribe Notes: Duality in General Programs}
\author{\Name}
\markboth{\CourseName\ -- \CourseSemester\ \CourseYear\ -- Lecture}{\CourseName\ -- \CourseSemester\ \CourseYear\ -- Lecture}
\pagestyle{myheadings}

\DeclarePairedDelimiter\ceil{\lceil}{\rceil}%
\DeclarePairedDelimiter\floor{\lfloor}{\rfloor}%
\DeclarePairedDelimiter\abs{\lvert}{\rvert}%
\DeclarePairedDelimiter\norm{\lVert}{\rVert}%
\newcommand{\E}{\mathbb{E}}
\newcommand{\R}{\mathbb{R}}
\newcommand{\st}{\text{s.t.}}
\newcommand{\Var}{\mathrm{Var}}
% From http://tex.stackexchange.com/questions/79434/
\newcommand\independent{\protect\mathpalette{\protect\independenthelper}{\perp}}
\def\independenthelper#1#2{\mathrel{\rlap{$#1#2$}\mkern2mu{#1#2}}}

\begin{document}

\maketitle

\section{Last Time, on Dragon Bal- Convex Optimization}

We discussed two explanations for duality in linear programs. Explanation \#1
is that we constrain $c$ in the objective $c^Tx$ to obtain a lower bound on the
optimal primal value. Alternatively, we can construct the Lagrangian and note that
the unconstrained Lagrangian is a lower bound. TODO(achald): This is a pretty
shit explanation.

\section{Lagrangian}

Consider any general minimization problem
\begin{alignat*}{2}
&\min_x &&f(x) \\
&\text{ subject to}&~&h_i(x) \leq 0, i = 1, \dots, m \\
&&&l_j(x) = 0, j = 1, \dots, r
\end{alignat*}

Let's define the Lagrangian, introducing variables $u \in \R^m, v \in \R^r$,
with $u \geq 0$.
\begin{align*}
L(x, u, v) = f(x) + \sum_{i=1}^m u_i h_i(x) + \sum_{j=1}^r v_j l_j(x)
\end{align*}

It turns out that for any $u \geq 0$ and any $v$, we have that
\begin{align*}
L(x, u, v) = f(x) \sum_{i=1}^m u_i \underbrace{h_i(x)}_{\leq 0} + \sum_{j=1}^r v_j \underbrace{l_j(x)}_{=0} \leq f(x)
\end{align*}

And so, we have that if $f^*$ be the primal optimal value and $C$ is the primal
feasible set, then
\begin{align*}
f^* \geq \min_{x \in C} L(x, u, v) \geq \min_x L(x, u, v) \triangleq g(u, v)
\end{align*}

This $g(u, v)$ is the \textbf{Lagrange dual function}, and it provides a lower
bound on the optimal value $f^*$ for any dual feasible $u, v$ (i.e. $u \geq 0$
and any $v$).

Generally, duality will provide us with a tight lower bound in the
\textit{convex} case, but generally will not in the \textit{non-convex} case.

\subsection{Example: Quadratic Program}
Consider a quadratic program where $Q \succ 0$:
\begin{alignat*}{2}
&\min_x& &\frac{1}{2} x^T Q x + c^T x \\
&\text{ subject to}&~&Ax = b, x \geq 0
\end{alignat*}

In this case, our Lagrangian is simply
\[ L(x, u, v) = \frac{1}{2} x^T Q x + c^T x - u^T x + v^T(Ax - b) \]

To compute the dual function $g(u, v) = \min_x L(x, u, v)$, we minimize the
Lagrangian above by taking the gradient with respect to $x$ and setting it equal
to zero, and we get that
\begin{align*}
x^* &= -Q^{-1}(c - u + A^T v) \\
\implies \min_x L(x, u, v) &= L(x^*, u, v) \\
    &= \frac{1}{2} (c - u + A^T v)^T Q^{-1}(c - u + A^T v) - (c - u + A^T v)^T Q^{-1}(c - u + A^T v) - b^T v \\
    &= -\frac{1}{2} (c - u + A^T v)^T Q^{-1}(c - u + A^T v) - b^T v
\end{align*}

What if, instead, we had the same QP as above, except $Q
\mathbin{\textcolor{red}{\succeq}} 0$ (i.e. Q is only positive
\textit{semi}-definite).  Then, if we try to minimize the Lagrangian above by
setting the gradient to 0, we get the following constraint at the optimum:
\begin{align}
Qx = -(c - u + A^T v) \label{eq:qp_semidefinite_zero_gradient}
\end{align}

Now, there are two cases:
\begin{enumerate}[(i)]
\item $c - u + A^T v \in \text{col}(Q)$. Then, we can use the pseudo-inverse
      $Q^{\dagger}$ of $Q$.
\item $c - u + A^T v \not\in \text{col}(Q)$, which implies that $c - u + A^T v$
      is not orthogonal to the null space of $Q$. Then, let
      \[ c - u + A^T v = z_1 + z_2, \]
      where $z_1 \in \text{col}(Q), z_2 \in \text{null}(Q), z_2 \neq 0$. But
      in this case, there is no $x$ that satisfies
      \cref{eq:qp_semidefinite_zero_gradient}, and so there is no unique
      minimizer $x^*$. But $L(x, u, v)$ is quadratic in $x$, so if there is no
      minimizer of $L(x, u, v)$ in $x$, the minimum must be $L(x, u, v) =
      -\infty$ (TODO: I think there's an alternative, easier to understand way
      of saying this.)
\end{enumerate}

So,
\begin{align*}
g(u, v) = \begin{cases}
-\frac{1}{2} (c - u + A^T v)^T Q^{\dagger} (c - u + A^Tv) - b^T v ~&\text{case (i)} \\
-\infty ~&\text{case (ii)}
\end{cases}
\end{align*}

\subsubsection{Aside: Pseudo-inverse}
For a general matrix $A \in \R^{n\times n}$, we can define the pseudo-inverse
$A^{\dagger}$ in terms of its Singular Vector Decomposition. Using SVD, we can
write $A$ as
\[ A = U D V^T \]

If $A$ was invertible, we can directly invert the decomposition above:
\[ A^{-1} = (U D V^T)^{-1} = (V^T)^{-1} D^{-1} U^{-1} = V D^{-1} U^T \]
where
\begin{align*}
D^{-1} = \begin{bmatrix}
\frac{1}{d_1} & 0 & \dots & 0 \\
0 & \frac{1}{d_2} & \dots & 0 \\
0 & 0 & \ddots & \vdots \\
0 & 0 & \dots & \frac{1}{d_n} \\
\end{bmatrix}
\end{align*}

If $A$ is not invertible, we're going to see that for $k = \text{rank}(A)$,
$d_{k+1} = d_{k+2} = \dots = d_n = 0$. In this case, we can construct the

\section{Convexity of the dual}

Even if the primal is non-convex, the dual problem is a convex optimizaiton
problem in $u, v$:
\begin{align*}
g(u, v) &= \min_x \{f(x) + \sum_{i=1}^m u_i h_i(x) + \sum_{j=1}^r v_j l_j(x) \} \\
        &= - \underbrace{\max_x \{-f(x) - \sum_{i=1}^m u_i h_i(x) - \sum_{j=1}^r v_j l_j(x) \}}_{\text{pointwise maximum of convex functions in } (u, v)}
\end{align*}

So why don't we just always write down the dual problem and solve it, if it's
convex? It turns out computing $g(u, v)$ is hard in and of itself, since it
involves a maximization over $x$, especially for non-convex problems. In other
words, we might not be able to write out $g(u, v)$ in the first place!

\section{Strong Duality}
We know that $f^* \geq g^*$, which is known as weak duality. In some problems, we
see that $f^* = g^*$. This is known as \textbf{strong duality}.

\textbf{Slater's condition}: If the primal is a convex problem, and there exists
at least one \textit{strictly} feasible $x \in \R^n$, then strong duality holds.

That is, for a general convex primal problem,
\begin{alignat*}{3}
&\min_x &&f(x) \\
&\text{subject to } && h_i(x) \geq 0 ~~~&&\text{for } 1 \leq i \leq m \\
&&& l_j(x) = 0 ~~~&&\text{for } 1 \leq j \leq r
\end{alignat*}

where $h_i(x)$ is convex and $l_j(x)$ is affine, and a strictly feasible $x$ is
an $x$ such that for every $i, h_i(x) < 0$ and for every
$j, l_j(x) = 0$.
\end{document}
